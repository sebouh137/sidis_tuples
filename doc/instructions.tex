\documentclass[english]{article}
%\documentclass[english,10pt,final,twocolumn]{article}

%\setlength{\mathindent}{0pt}
\usepackage[T1]{fontenc}
\usepackage[latin9]{inputenc}
\usepackage{pdfpages}
\usepackage{amstext}
\usepackage{graphicx}
\usepackage{gensymb}
\usepackage{amssymb}
\usepackage{placeins}
\usepackage{subfigure}
\usepackage[title]{appendix}
\makeatletter
\usepackage{float}
\newcommand{\GeVsq}[0]{$({\rm GeV}\!/\!c)^2$}
%\usepackage{subcaption}
\usepackage{blindtext}
\usepackage{titlesec}

%\titleformat{\subsection}
%  {\normalfont\scshape}{\thesubsection}{1em}{}
\titleformat{\section}
  {\bfseries\large}{\thesection}{1em}{}
\titleformat{\subsection}
  {\bfseries}{\thesubsection}{1em}{}
  \titleformat{\appendix}
  {\bfseries\large}{\thesection}{1em}{}


%%TeXShop comment:
% !TEX pdfSinglePage

\newcommand{\lyxmathsym}[1]{\ifmmode\begingroup\def\b@ld{bold}
  \text{\ifx\math@version\b@ld\bfseries\fi#1}\endgroup\else#1\fi}
\usepackage{braket}
%\usepackage{mathtools}
\usepackage{mathrsfs}
\usepackage{slashed}
\usepackage{bm}
\usepackage{datetime}
\usepackage{siunitx}

%in order to use align
\usepackage{amsmath}

\DeclareSIUnit[number-unit-product = {}]\clight{c}
\DeclareSIUnit\eVperc{\eV\per\clight}
\DeclareSIUnit\GeVpercs{\giga\eV\squared\per\clight\squared}
\DeclareSIUnit\MeVpercs{\mega\eV\per\clight\squared}
\sisetup{per-mode = symbol}
%\journal{Physics Letters B}
%\biboptions{numbers,sort&compress}
\usepackage[left=1.6cm,right=1.04cm,top=1.65cm,bottom=1.65cm,columnsep=25pt]{geometry}
\usepackage{lineno}
%\linenumbers
\usepackage{hyperref}  % hyper-links 
\hypersetup{breaklinks = true, colorlinks = true, allcolors = magenta}
\hyphenation{ALPGEN}
\hyphenation{EVTGEN}
\hyphenation{PYTHIA}

\usepackage{setspace}
\usepackage{graphicx}
\usepackage{wrapfig}

%\newcommand{\code}[1]{\begin{verbatim}#1\end{verbatim}}
\newcommand{\code}[1]{\texttt{#1}}
\makeatother

\usepackage{babel}
\begin{document}


\title{
Analysis Documentation for Clas12 Sidis Tuple Maker
}

\author{Sebouh Paul}
\date{\today}

\maketitle
\tableofcontents
\newpage

\section{How to make tuples}
Example:


\begin{verbatim}clas12root -l -b -q src/SidisTuples.C+ 
--in=/work/clas12/rg-a/montecarlo/fall2018/torus-1/clasdis/nobg/DIS_pass1_997_1002.hipo 
 --out=mc_electrons.root --isMC --skipEvents=600000 --N=300000 --includeElectrons\end{verbatim}

This creates opens the file \code{/work/clas12/rg-a/montecarlo/fall2018/torus-1/clasdis/nobg/DIS\_pass1\_997\_1002.hipo}, and outputs a file \code{mc\_electrons.root}.  The \code{--isMC} flag tells the program that the input file is from a Monte-Carlo simulation.  The optional \code{--skipEvents} and \code{--N} arguments tells the program to skip the first 600k events, and then processes the next 300k events.  The output contains only the electrons tuple, as specified by the \code{--includeElectrons} flag.  The hadrons, dihadrons and dipions tuples can be created by using the \code{--includeHadrons}, \code{--includeDihadrons}, and \code{--includeDipions} flags respectively.  If no flag for including a tuple in the output is provided, then the program will terminate without processing any data.  
 

\section{Trees generated}
\begin{itemize}
\item[electrons] all electrons that pass fiducial, PID, and DIS cuts.  There may be multiple entries per event if there are multiple DIS electrons in the event.
\item[hadrons] DIS electron + hadron.  There may be multiple entries per event if there are multiple hadrons in the event.  
\item[dihadrons] DIS electron + leading pion + another hadron (which may or may not be a pion)
\item[dipions] DIS electron + pion + second pion.  Neither pion is required to be leading.  
\end{itemize}

\end{document}